実世界では,蜂,アリなどの昆虫の簡単な行動メカニズムによる複雑な群れ行動がよく観察される.
また,大きな交差点などにおいて,人間は密度が高くても,会話なしで,
ぶつからないようにスムーズに対面歩行ができる.
その他にも自動車の交通流や生物細胞など,自己駆動粒子の集団運動は至るところで観察され,
その発生メカニズムには興味深いものがある.
しかし人間の集団歩行を例に取れば分かるとおり,その行動メカニズムすなわち行動アルゴリズム
は必ずしも明確ではない\cite{murakami2021}.


我々は,走行ロボットを使って,様々な自律走行アルゴリズムで走行実験を行い,
その隊列走行など,群ロボットの自己組織化現象について研究を行っている.
群衆の対面歩行におけるレーン形成などと比較することができれば,その生成メカニズムを
推測する手がかりになることが期待できる.

先行研究\cite{li2020}では,我々は双曲線関数により,
距離センサとモーターの出力が直接関連する感覚運動写像(省略名:SMM)ロボットを使って,
擬楕円コースで8台のロボットの対面走行の実験を行った.
結果として,コース幅の増加に従って,流量が増加して,
コース幅がロボットサイズの約1.8倍程度から自律走行が安定することを確認した.
初期配置と初期向きを問わず,一定時間走行後,
すべてのロボットが同じ方向になって,1方向走行流が観察された.
理由として,距離情報だけでは, 向きを判断できないため考えている.
また,3つの距離データで障害物の数を認識できず,渋滞が起こることの原因になると考える.

本研究では,長時間対面走行ができ,流量を増やせるようにするため,
カメラを使って,中間層1層のニューラルネットワーク(省略名:NN)により1次元画像認識アルゴリズムを開発した.
コースに色をつけ,感覚運動写像とニューラルネットワーク2種類のアルゴリズムでコース幅1mの8台ロボットの
対面走行実験を行った.

モーションキャプチャによってロボットの走行軌道を観測し,その流量を求めた.
感覚運動写像より, ニューラルネットワークの方が壁や障害物(他のロボット)をスムーズに避けることができる
ことを観察した.流量も約1.5倍になると確認した.

また教師データを作成する際に,モーター駆動の変化量を相対的に極端にする(大きなハンドル)方が
ニューラルネットワークによる自律走行性能が向上することが明らかとなった.


