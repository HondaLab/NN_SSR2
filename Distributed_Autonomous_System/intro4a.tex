実世界で,蜂,アリなどの昆虫が簡単な行動メカニズムによって,複雑な群れ行為がよく観察される.
また,大きな交差点などにおいて,人間は密度が高くても,会話なしで,
ぶつからないようにスムーズに対面歩行ができる.
我々は,走行ロボットを使って,様々な自律走行アルゴリズムで対面走行実験を行って,
その対面走行など,自己組織化のメカニズムを解明することに目指す.

先行研究\cite{li}では,我々は双曲線関数により,
距離センサとモーターの出力が直接関連する感覚運動写像(省略名:SMM)ロボットを使って,
擬楕円コースで8台のロボットの対面走行の実験を行った.
結果として,コース幅の増加に従って,流量が増加して,
コース幅が56cmから安定になると確認した.
初期配置と初期向きを問わず,一定時間結果後,
すべてのロボットが同じ方向になって,1方向走行流が観察された.
理由として,距離データで向きを判断できないと考えている.
それに,3つの距離データで障害物の数を認識できず,
渋滞が起こることの原因になると考える.

本研究では,長時間対面走行ができ,流量を増やせるようにするため,
カメラを使って,中間層1層のニューラルネットワーク(省略名:NN)により1次元画像認識アルゴリズムを開発した.
コースに色をつけ,感覚運動写像とニューラルネットワーク2種類のアルゴリズムでコース幅1mの8台ロボットの対面走行を実験した.

結果として,ラジコンする時,ソースコード上曲がる値を拡大,
直接後退機能をつければ,ニューラルネットワークより自律走行性能が向上するとしった.
一方,方向転換のことをロボットに教えてないまま,1方向走行流になりにくく,感覚運動写像より,
ニューラルネットワークの方が壁や障害物(他のロボット)をスムーズに避けることができると観察した,流量も1.5になると確認した.

