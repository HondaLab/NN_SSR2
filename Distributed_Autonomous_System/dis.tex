1次元画像データ認識ニューラルネットワークにより8台ロボットが縞模様のコースで
対面走行ができると確認した.
感覚運動写像と比べて,方向転換を教えなくても,対面走行を維持する能力が高い.

対面走行維持できるニューラルネットワークの学習結果は,著者が何十回と練習して,経験を踏まえ,
ロボットを操縦する時,遠くから曲がる,近づいて曲がる,真正面にロボットや壁がある時後退することを
わざわざ人間の意識持ってロボットに教えて収集した教師データの学習結果である.
著者以外の人がラジコン操縦して収集した教師データを学習しても,必ずしも本論文で示した自律走行の質
が再現できるとは限らない.
教師データが備えるべき客観的性質を定量的に測定することが今後の課題である.
このことは人間が群衆の中で対面歩行する際の具体的アルゴリズムを考察する上で貴重な示唆を与えると
考えられる.

また,コースの壁の代わりに,別のもの(ダンボール,雑誌,サンダルなど)を配置して,
収集した教師データの学習結果では自律走行が困難であることが観察された.
その原因として,画像の列ずつ足し算してつくった1次元画像データが障害物と通路の特徴を失って,
区別できなくなると考えられる.

今後の展望として,さまざまな教師データに対する自律走行実験を行い,
特徴を分析して,どのような教師データがあればスムーズな対面走行を維持できるかを明らかにしたい.
そのためには教師データの質の評価方法を開発する必要がある.
また,画像エントロピーで環境の複雑度を評価して1次元画像データの限界を解明したいと考えている.

